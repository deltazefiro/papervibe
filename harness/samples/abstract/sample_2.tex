\documentclass{article}
\usepackage{amsmath}
\usepackage{amssymb}

\title{NeuroVision: A Novel Framework for Neural Image Synthesis}
\author{Anonymous Authors}

\begin{document}

\maketitle

\begin{abstract}
We present NeuroVision, a novel framework for neural image synthesis that achieves state-of-the-art results on multiple benchmarks. Our approach combines diffusion models with a hierarchical latent space representation that enables fine-grained control over generated images. The key insight behind NeuroVision is the use of a multi-scale attention mechanism that progressively refines image details from coarse to fine. We train our model on a dataset of 500 million image-text pairs and demonstrate that it can generate photorealistic images at resolutions up to 4K. Extensive experiments show that NeuroVision outperforms existing methods on FID (3.2 vs. previous best 4.1), CLIP score (32.5 vs. 28.7), and human preference studies (78\% preferred over baselines). We also introduce a new editing interface that allows users to modify specific regions of generated images through natural language instructions. Our code and pretrained models will be released upon acceptance.
\end{abstract}

\section{Introduction}

Image synthesis has become a fundamental task in computer vision, with applications ranging from content creation to data augmentation.

\section{Method}

Our framework consists of three main components: (1) a hierarchical encoder that maps images to a multi-scale latent space, (2) a diffusion-based decoder that generates images from latent codes, and (3) a text-to-latent module that conditions the generation on natural language descriptions.

\subsection{Hierarchical Latent Space}

Let $z = (z_1, z_2, ..., z_L)$ denote the hierarchical latent representation, where $z_l \in \mathbb{R}^{d_l}$ is the latent code at level $l$. The encoder $E_\theta$ maps an input image $x$ to this representation:

\begin{equation}
z = E_\theta(x)
\end{equation}

\end{document}
