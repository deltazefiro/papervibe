\documentclass{article}
\usepackage{amsmath}
\usepackage{amssymb}

\title{Understanding Large Language Models: A Survey}
\author{Anonymous Authors}

\begin{document}

\maketitle

\begin{abstract}
Large language models (LLMs) have recently emerged as a transformative technology in natural language processing. These models, typically built on transformer architectures with billions of parameters, demonstrate remarkable capabilities in text generation, question answering, and reasoning tasks. In this survey, we provide a comprehensive overview of the development of LLMs, from early neural language models to modern architectures like GPT-4 and Claude. We analyze the key technical innovations that enabled scaling to billions of parameters, including attention mechanisms, positional encodings, and efficient training techniques. Furthermore, we discuss the emergent capabilities observed in these models, such as in-context learning and chain-of-thought reasoning, which were not explicitly programmed but arose from scale. We also examine the challenges facing the field, including computational costs, hallucinations, and alignment with human values. Finally, we outline promising future directions, including more efficient architectures, improved reasoning capabilities, and methods for better controllability.
\end{abstract}

\section{Introduction}

The field of natural language processing has witnessed remarkable advances with the introduction of large language models. These models have fundamentally changed how we approach language understanding and generation tasks.

\section{Background}

Neural language models have evolved significantly over the past decade. Early approaches used recurrent neural networks, but modern models employ transformer architectures that enable parallel processing and better capture of long-range dependencies.

\end{document}
